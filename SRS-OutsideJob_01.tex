\documentclass[12pt,a4paper,oneside]{article}
\usepackage[utf8]{inputenc}
\usepackage[T1]{fontenc}
\usepackage[polish]{babel}
\usepackage[margin=1in]{geometry}
\usepackage{listings}
\usepackage{xcolor}
\usepackage{graphicx}
\usepackage{hyperref}
\usepackage{fancyhdr}
\usepackage{array}
\usepackage{tabularx}
\usepackage{amsfonts}
\usepackage{amssymb}
\usepackage{longtable}

% Konfiguracja listingów dla Kotlin
\lstdefinelanguage{Kotlin}{
  comment=[l]{//},
  commentstyle=\color{gray}\ttfamily,
  emph={filter, map, fold, forEach, println, class, object, fun, val, var, if, else, when, for, in, while, return, true, false, null},
  emphstyle=\color{blue}\bfseries,
  identifierstyle=\color{black},
  keywords={class, fun, val, var, object, interface, enum, package, import, as, is, in, where},
  keywordstyle=\color{blue}\bfseries,
  morecomment=[s]{/*}{*/},
  morestring=[b]",
  morestring=[s]{"""}{"""},
  ndkeywords={@Composable, @Preview, @Entity, @PrimaryKey},
  ndkeywordstyle=\color{purple}\bfseries,
  sensitive=true,
  stringstyle=\color{red},
  breaklines=true,
  showstringspaces=false,
  tabsize=2,
}

\lstset{
    language=Kotlin,
    basicstyle=\ttfamily\small,
    backgroundcolor=\color{gray!10},
    frame=single,
    framexleftmargin=5pt,
    frameright=true,
    framebottom=true,
    framextopsep=5pt,
    framesep=5pt,
    numbers=left,
    numberstyle=\tiny\color{gray},
    xleftmargin=20pt,
}

% Header i Footer
\pagestyle{fancy}
\fancyhf{}
\rhead{OutsideJob -- Specyfikacja Wymagań}
\lhead{Android -- Aplikacja Terenowa}
\cfoot{\thepage}

\hypersetup{
    colorlinks=true,
    linkcolor=blue,
    filecolor=blue,
    urlcolor=blue,
    bookmarksnumbered=true,
}

\title{\Large\textbf{OutsideJob\\ Specyfikacja Wymagań Funkcjonalnych i Niefunkcjonalnych\\ Aplikacja do Zarządzania Raportami Terenowymi}}
\author{Dokumentacja Projektowa}
\date{\today}

\begin{document}

\maketitle

\tableofcontents
\newpage

% ============================================================================
\section{Wstęp}
% ============================================================================

\subsection{Cel projektu}

OutsideJob to platforma mobilna przeznaczona dla firm posiadających floty serwisantów i pracowników terenowych. Celem aplikacji jest umożliwienie prostego i jednolitego przeprowadzania raportów z podjętych zleceń (zadań) bezpośrednio z terenu, z możliwością offline-first i automatycznej synchronizacji po powrocie do internetu.

\subsection{Zakres aplikacji}

\begin{itemize}
    \item Zarządzanie listą przydzielonych zadań (Tasks)
    \item Tworzenie szczegółowych raportów z wykonanych zleceń
    \item Wygenerowanie raportów w formatach wymaganych przez klienta (PDF, JSON, Excel itp.)
    \item Praca offline z możliwością synchronizacji danych
    \item Filtowanie i wyszukiwanie zadań
    \item Zarządzanie załącznikami (zdjęcia, dokumenty)
    \item Kontrola dostępu poprzez logowanie
\end{itemize}

\subsection{Adresaci i interesariusze}

\begin{itemize}
    \item \textbf{Pracownicy terenowi} (serwisanci, elektrycy, hydraulicy itp.) -- główni użytkownicy aplikacji
    \item \textbf{Kierownicy pracowników} -- mogą przydzielać zadania
    \item \textbf{Administratorzy OutsideJob} -- zarządzają użytkownikami i systemem
    \item \textbf{Firmy-klienci} -- otrzymują raporty w wymaganych formatach
\end{itemize}

\subsection{Dokumenty referencyjne}

\begin{itemize}
    \item Architektura Synchronizacji Offline (osobny dokument)
    \item API REST (specyfikacja endpoints)
    \item Wytyczne UX/UI (wireframy/mockupy)
\end{itemize}

% ============================================================================
\section{Wymagania funkcjonalne}
% ============================================================================

\subsection{REQ-F1: Autentykacja i logowanie}

\textbf{Opis:} Użytkownik musi się zalogować do aplikacji za pomocą adresu e-mail i hasła otrzymanego od administratora OutsideJob.

\begin{itemize}
    \item REQ-F1.1: Ekran logowania zawiera pola: ``Email'', ``Hasło''
    \item REQ-F1.2: Po wpisaniu niepoprawnych danych wyświetlane jest komunikat błędu
    \item REQ-F1.3: Po zalogowaniu aplikacja synchronizuje listę zadań z serwerem
\end{itemize}

\textbf{Priorytet:} KRYTYCZNE

\textbf{Aktor:} Pracownik terenowy

---

\subsection{REQ-F2: Zapamiętanie loginu i hasła}

\textbf{Opis:} Użytkownik przy pierwszym logowaniu ma możliwość zapamiętania loginu i hasła dla automatycznego zalogowania przy kolejnych uruchomieniach aplikacji.

\begin{itemize}
    \item REQ-F2.1: Checkbox ``Zapamiętaj mnie'' dostępny na ekranie logowania
    \item REQ-F2.2: Dane uwierzytelniające przechowywane lokalnie w \textbf{Android Keystore} (bezpieczne)
    \item REQ-F2.3: Przy wylogowaniu dane usuwane z lokalnego magazynu
    \item REQ-F2.4: Logowanie na innym urządzeniu powoduje wylogowanie na poprzednim
\end{itemize}

\textbf{Priorytet:} WYSOKI

\textbf{Aktor:} Pracownik terenowy

---

\subsection{REQ-F3: Lista zadań (Tasks) po logowaniu}

\textbf{Opis:} Po zalogowaniu użytkownik otrzymuje zsynchronizowaną listę zadań przydzielonych do niego.

\begin{itemize}
    \item REQ-F3.1: Lista zadań pobierana z serwera przy logowaniu
    \item REQ-F3.2: Zadania sortowane po dacie utworzenia (najnowsze u góry)
    \item REQ-F3.3: Każde zadanie wyświetla: tytuł, krótki opis, status, datę terminu
    \item REQ-F3.4: Ikona statusu (kolor/ikonka) wskazuje stan: TODO, IN\_PROGRESS, COMPLETED, SUSPENDED, OVERDUE
    \item REQ-F3.5: Możliwość kliknięcia w zadanie dla wyświetlenia całego zadania z przyciskami rozpocznij dojazd rozpocznij prace
\end{itemize}

\textbf{Priorytet:} KRYTYCZNE

\textbf{Aktor:} Pracownik terenowy

---

\subsection{REQ-F4: Filtrowanie listy zadań}

\textbf{Opis:} Użytkownik ma możliwość filtrowania zadań między kategoriami.

\begin{itemize}
    \item REQ-F4.1: Filtry dostępne w górnym menu (po prawej): ALL, TODO, IN\_PROGRESS, COMPLETED, SUSPENDED, OVERDUE
    \item REQ-F4.2: Wybranie filtru powoduje przeliczenie listy lokalnie
    \item REQ-F4.3: Aktualny filtr wyświetlany w nagłówku ekranu
    \item REQ-F4.4: Preferencja ostatniego wybranego filtru przechowywana lokalnie
    \item REQ-F4.5: Przy każdorazowym uruchomieniu aplikacji pracownik widzi filtr TODO
    \item REQ-F4.6: Zadanie które nigdy nie wyło wyśweitlone (pracownik nigdy jeszce w niego nie kliknął nie widział szczegółów wyróżnia się kolorem jest jako new) ale stan pozostaje TODO
\end{itemize}

\textbf{Priorytet:} WYSOKI

\textbf{Aktor:} Pracownik terenowy

---

\subsection{REQ-F5: Szczegóły zadania}

\textbf{Opis:} Po wybraniu zadania z listy użytkownik widzi ekran ze szczegółami i możliwościami wykonania.

\begin{itemize}
    \item REQ-F5.1: Ekran zawiera: tytuł, pełny opis, termin, przydzielony pracownik, lokalizacja, załączniki
    \item REQ-F5.2: Przycisk ``Rozpocznij dojazd'' -- zmienia status na IN\_TRANSIT
    \item REQ-F5.3: Przycisk ``Rozpocznij pracę'' -- zmienia status na IN\_PROGRESS
    \item REQ-F5.4: Przycisk ``Wstrzymaj'' -- zmienia status na SUSPENDED (wymaga opisu) tylko w stanie IN\_PROGRESS lub IN\_TRANSIT
    \item REQ-F5.5: Przycisk ``Wznów'' -- powrót do IN\_PROGRESS (dostępny tylko ze stanu SUSPENDED)
\end{itemize}

\textbf{Priorytet:} KRYTYCZNE

\textbf{Aktor:} Pracownik terenowy

---

\subsection{REQ-F6: Rozpoczęcie dojazdu}

\textbf{Opis:} Pracownik może rozpocząć dojazd do lokalizacji zadania, zmienia status na IN\_TRANSIT.

\begin{itemize}
    \item REQ-F6.1: Przycisk ``Rozpocznij dojazd'' zmienia status z TODO na IN\_TRANSIT
\end{itemize}

\textbf{Priorytet:} WYSOKI

\textbf{Aktor:} Pracownik terenowy

---

\subsection{REQ-F7: Rozpoczęcie pracy nad zadaniem}

\textbf{Opis:} Pracownik rozpoczyna pracę nad zadaniem, zmienia status na IN\_PROGRESS, otwiera się checklist zadań.

\begin{itemize}
    \item REQ-F7.1: Przycisk ``Rozpocznij pracę'' zmienia status na IN\_PROGRESS
    \item REQ-F7.2: Wyświetla się lista podzadań (cały formularz raportowania) z checkboxami
    \item REQ-F7.3: Każde podzadanie może mieć opis, typ (checkbox, tekstowe pole, zdjęcie)
    \item REQ-F7.4: Jedno kliknięcie na podzadanie otwiera klawiaturę dla szybkiego wprowadzenia tekstu
    \item REQ-F7.5: Pracownik nie może mieć więcej niż jedno zlecenie o statusie IN\_PROGRESS oraz IN\_TRANSIT
\end{itemize}

\textbf{Priorytet:} KRYTYCZNE

\textbf{Aktor:} Pracownik terenowy

---

\subsection{REQ-F8: Wstrzymanie zadania (SUSPENDED)}

\textbf{Opis:} Pracownik może wstrzymać zadanie (np. oczekiwanie na części, przerwa), wymaga uzasadnienia.

\begin{itemize}
    \item REQ-F8.1: Przycisk ``Wstrzymaj'' dostępny ze stanu IN\_PROGRESS lub IN\_TRANSIT
    \item REQ-F8.2: Formularz zawiera pola dostosowane do konkretnego klienta (np. wymiana przedmiotów, wykonanie zdjęć)
    \item REQ-F8.3: Wyświetla się dialog z polem tekstowym ``Powód wstrzymania''
    \item REQ-F8.4: Pole obowiązkowe -- bez powodu nie można wstrzymać
    \item REQ-F8.5: Status zmienia się na SUSPENDED
    \item REQ-F8.6: Przycisk ``Wznów pracę'' powraca do IN\_PROGRESS
    \item REQ-F8.7: Historia wstrzymań przechowywana w raporcie
\end{itemize}

\textbf{Priorytet:} WYSOKI

\textbf{Aktor:} Pracownik terenowy

---

\subsection{REQ-F9: Rozliczenie pracy (Completion)}

\textbf{Opis:} Pracownik kończy pracę, podaje ostateczny raport, zmienia status na COMPLETED.

\begin{itemize}
    \item REQ-F9.1: Przycisk ``Zakończ pracę'' sprawdza czy wszystkie obowiązkowe pola zostały uzupełnione jeżeli tak to zamyka zlecenie i zmienia status na COMPLETED
    \item REQ-F9.2: PO zakończeniu pracy brak mozliwości edycji jakichkolwiek pól
    \item REQ-F9.3: Po potwierdzeniu -- upload raportów/zdjęć i zmiana statusu na COMPLETED
    \item REQ-F9.4: Zadanie trafia do historii, nie można go edytować
\end{itemize}

\textbf{Priorytet:} KRYTYCZNE

\textbf{Aktor:} Pracownik terenowy

---

\subsection{REQ-F10: Dodawanie zadania (jeśli wymaga klient)}

\textbf{Opis:} Pracownik (lub kierownik) może dodać nowe zadanie do swojej listy.

\begin{itemize}
    \item REQ-F10.1: Zakładka ``+ Dodaj zadanie'' w navigation drawer
    \item REQ-F10.2: Formularz zawiera: tytuł, opis, lokalizacja, termin, przydzielenie do pracownika
    \item REQ-F10.3: Jeśli pracownik -- zawsze przydzielane sobie
    \item REQ-F10.4: Jeśli kierownik -- może wybrać pracownika z listy
    \item REQ-F10.5: Zadanie tworzone TYLKO online
    \item REQ-F10.6: Po utworzeniu -- serwer generuje ID, synchronizuje się na urządzenia
\end{itemize}

\textbf{Priorytet:} ŚREDNI

\textbf{Aktor:} Pracownik terenowy, Kierownik

---

\subsection{REQ-F11: Logout}

\textbf{Opis:} Użytkownik może się wylogować z aplikacji.

\begin{itemize}
    \item REQ-F11.1: Opcja ``Wyloguj'' w navigation drawer (dół, po lewej)
    \item REQ-F11.2: Po wylogowaniu -- usuwane zapamiętane dane logowania (jeśli były)
    \item REQ-F11.3: Powrót do ekranu logowania
    \item REQ-F11.4: Lokalne dane zadań pozostają (offline-first), ale logowanie na innym urządzeniu tę sesję wyłącza
\end{itemize}

\textbf{Priorytet:} WYSOKI

\textbf{Aktor:} Pracownik terenowy

---

\subsection{REQ-F12: Dodawanie załączników (Zdjęcia, Notatki)}

\textbf{Opis:} Pracownik może dodawać zdjęcia, notatki głosowe, dokumenty do zadania.

\begin{itemize}
    \item REQ-F12.1: Przycisk ``+ Dodaj zdjęcie'' w ekranie pracy
    \item REQ-F12.2: Możliwość robienia zdjęcia bezpośrednio lub wybrania z galerii
    \item REQ-F12.3: Zdjęcia przechowywane lokalnie, uploaded przy zatwierdzeniu zadania
    \item REQ-F12.4: Pole notatek dla każdego podzadania
    \item REQ-F12.5: Notatki głosowe (opcjonalnie) -- nagranie i transkrypcja
\end{itemize}

\textbf{Priorytet:} WYSOKI

\textbf{Aktor:} Pracownik terenowy

---

\subsection{REQ-F13: Interfejs użytkownika (UI Layout)}

\textbf{Opis:} Struktura ekranów i nawigacji aplikacji.

\begin{itemize}
    \item REQ-F13.1: Navigation Drawer po lewej stronie (``Tasks'', ``+ Add Task'', ``Wyloguj'')
    \item REQ-F13.2: Góry ekran - toolbar z: ikoną menu (hamburger), tytułem, ikoną filtrów
    \item REQ-F13.3: Menu filtrów po prawej w toolbarze (ALL, TODO, IN\_PROGRESS, COMPLETED, SUSPENDED, OVERDUE)
    \item REQ-F13.4: Bottom sheet z szybkimi akcjami (jeśli potrzebne)
    \item REQ-F13.5: FAB (Floating Action Button) ``+'' dla szybkiego dodania zadania
\end{itemize}

\textbf{Priorytet:} ŚREDNI

\textbf{Aktor:} Pracownik terenowy

---

\subsection{REQ-F14: Synchronizacja danych}

\textbf{Opis:} Dane synchronizują się automatycznie i ręcznie.

\begin{itemize}
    \item REQ-F14.1: Automatyczna synchronizacja co 15 minut (jeśli są zmiany)
    \item REQ-F14.2: Ręczna synchronizacja -- przciągnięcie listy w dół (pull-to-refresh)
    \item REQ-F14.3: Po powrocie internetu -- automatyczna synchronizacja pending zmian
    \item REQ-F14.4: Powiadomienie o sukcesie/błędzie synchronizacji
    \item REQ-F14.5: Brak konfliktów (zadania zakończone niezmienne)
\end{itemize}

\textbf{Priorytet:} KRYTYCZNE

\textbf{Aktor:} System

---

% ============================================================================
\section{Wymagania niefunkcjonalne}
% ============================================================================

\subsection{NFR-1: Wydajność i skalowalność}

\begin{itemize}
    \item NFR-1.1: Minimalna obsługa 50 użytkowników równolegle
    \item NFR-1.2: Czas ładowania listy zadań: < 2 sekundy (gdy online)
    \item NFR-1.3: Czas otwarcia szczegółów zadania: < 1 sekunda
    \item NFR-1.4: Maksymalnie 100 zadań na liście bez lag'ów
    \item NFR-1.5: Architektura skalowalna do 1000+ użytkowników w przyszłości
\end{itemize}

---

\subsection{NFR-2: Bezpieczeństwo}

\begin{itemize}
    \item NFR-2.1: Cała komunikacja przez HTTPS (TLS 1.2+)
    \item NFR-2.2: Hasła przechowywane haszowane (bcrypt/Argon2) na serwerze
    \item NFR-2.3: Dane zapamiętane na urządzeniu w Android Keystore (szyfrowane)
    \item NFR-2.4: Logowanie z poziomu administratora możliwe dla audytu
    \item NFR-2.5: Raporty zawierają dane pełne tylko dla autoryzowanych osób
\end{itemize}

---

\subsection{NFR-3: Dostępność i niezawodność}

\begin{itemize}
    \item NFR-3.1: Wysoka dostępność (HA) 24/7 -- SLA 99.5\%
    \item NFR-3.2: Backend w architekturze multi-region (ang. failover)
    \item NFR-3.3: Backup danych co 6 godzin
    \item NFR-3.4: Recovery Time Objective (RTO): < 1 godzina
    \item NFR-3.5: Recovery Point Objective (RPO): < 15 minut
\end{itemize}

---

\subsection{NFR-4: Offline-first}

\begin{itemize}
    \item NFR-4.1: Wszystkie dane przechowywane lokalnie w Room (SQLite)
    \item NFR-4.2: Pracownik może pracować 100\% offline
    \item NFR-4.3: Synchronizacja dwukierunkowa (push/pull)
    \item NFR-4.4: Brak konfliktów dla zamkniętych zadań
    \item NFR-4.5: Offline-first = dane najpierw lokalnie, potem serwer
\end{itemize}

---

\subsection{NFR-5: Kompatybilność}

\begin{itemize}
    \item NFR-5.1: Android 7.0 (API 24) minimum
    \item NFR-5.2: Obsługa ekranów od 4.5'' do 6.7''
    \item NFR-5.3: Orientacja portrait i landscape
    \item NFR-5.4: Obsługa różnych DPI (mdpi, hdpi, xhdpi, xxhdpi)
    \item NFR-5.5: Testowanie na minimalnie 5 popularnychmodelach urządzeń
\end{itemize}

---

\subsection{NFR-6: Responsywność i UX}

\begin{itemize}
    \item NFR-6.1: Szybka reakcja na interakcje użytkownika (< 300ms)
    \item NFR-6.2: Animacje płynne (60 FPS)
    \item NFR-6.3: Material Design 3 lub nowszy
    \item NFR-6.4: Dostępność (A11y) -- obsługa czytników ekranu
    \item NFR-6.5: Kontrast tekstu minimum 4.5:1
\end{itemize}

---

\subsection{NFR-7: Observability i monitoring}

\begin{itemize}
    \item NFR-7.1: Centralne logowanie (np. ELK, Datadog)
    \item NFR-7.2: Monitoring wydajności backend'u (APM)
    \item NFR-7.3: Alerty na błędy i anomalie
    \item NFR-7.4: Dashboards dla: liczba aktywnych użytkowników, czas odpowiedzi API, błędy
\end{itemize}

---

% ============================================================================
\section{Przypadki użycia (Use Cases)}
% ============================================================================

\subsection{UC-1: Zalogowanie pracownika}

\textbf{Aktor:} Pracownik terenowy

\textbf{Warunki początkowe:} Aplikacja zainstalowana, użytkownik posiada e-mail i hasło od administratora

\textbf{Główny przepływ:}

\begin{enumerate}
    \item Pracownik uruchamia aplikację
    \item System wyświetla ekran logowania
    \item Pracownik wpisuje e-mail i hasło
    \item Pracownik opcjonalnie zaznacza ``Zapamiętaj mnie''
    \item Pracownik klika ``Zaloguj''
    \item System łączy się z serwerem i weryfikuje dane
    \item Po sukcesie -- pobiera listę zadań
    \item Aplikacja wyświetla listę Tasks
\end{enumerate}

\textbf{Alternatywne przepływy:}

\begin{itemize}
    \item A1: Niepoprawne hasło -- wyświetlany komunikat błędu, pracownik może spróbować ponownie
    \item A2: Brak internetu -- jeśli były poprzednie dane zapamiętane, system zaloguje offline
    \item A3: E-mail nie istnieje -- komunikat ``Użytkownik nie znaleziony''
\end{enumerate}

\textbf{Warunki końcowe:} Pracownik zalogowany, lista zadań widoczna

---

\subsection{UC-2: Przeglądanie listy zadań}

\textbf{Aktor:} Pracownik terenowy

\textbf{Warunki początkowe:} Pracownik zalogowany

\textbf{Główny przepływ:}

\begin{enumerate}
    \item System wyświetla listę wszystkich przydzielonych zadań
    \item Pracownik widzi: tytuł, krótki opis, status, datę terminu
    \item Pracownik może przewijać listę
    \item Pracownik klika na jedno z zadań
    \item System otwiera szczegóły zadania
\end{enumerate}

\textbf{Alternatywne przepływy:}

\begin{itemize}
    \item A1: Brak zadań -- wyświetlany komunikat ``Brak przydzielonych zadań''
    \item A2: Pracownik filtruje zadania (TODO, COMPLETED, SUSPENDED)
    \item A3: Pull-to-refresh -- ręczna synchronizacja z serwerem
\end{itemize}

\textbf{Warunki końcowe:} Lista zadań wyświetlona lub szczegóły zadania otwarte

---

\subsection{UC-3: Rozpoczęcie pracy nad zadaniem}

\textbf{Aktor:} Pracownik terenowy

\textbf{Warunki początkowe:} Pracownik wybrał zadanie ze statusem TODO lub IN\_TRANSIT

\textbf{Główny przepływ:}

\begin{enumerate}
    \item System wyświetla szczegóły zadania
    \item Pracownik klika ``Rozpocznij pracę''
    \item Status zmienia się na IN\_PROGRESS
    \item System wyświetla listę podzadań (checklist)
    \item Pracownik wykonuje podzadania, zaznacza checkboxy
    \item Pracownik dodaje notatki/zdjęcia
    \item Po zakończeniu -- pracownik klika ``Zakończ pracę''
\end{enumerate}

\textbf{Alternatywne przepływy:}

\begin{itemize}
    \item A1: Pracownik wstrzymuje zadanie -- przycisk ``Wstrzymaj'' z uzasadnieniem
    \item A2: Pracownik pracuje offline -- zmiany przechowywane lokalnie
\end{itemize}

\textbf{Warunki końcowe:} Status zmieniony na IN\_PROGRESS, lub SUSPENDED, lub COMPLETED

---

\subsection{UC-4: Wstrzymanie zadania}

\textbf{Aktor:} Pracownik terenowy

\textbf{Warunki początkowe:} Pracownik w stanie IN\_PROGRESS

\textbf{Główny przepływ:}

\begin{enumerate}
    \item Pracownik klika ``Wstrzymaj''
    \item System wyświetla dialog z polem ``Powód wstrzymania''
    \item Pracownik wpisuje powód (obowiązkowe)
    \item Pracownik klika ``Wstrzymaj''
    \item Status zmienia się na SUSPENDED
    \item System zapisuje zmianę lokalnie, odkłada do synchronizacji
\end{enumerate}

\textbf{Warunki końcowe:} Status SUSPENDED, powód wstrzymania zapisany

---

\subsection{UC-5: Rozliczenie pracy}

\textbf{Aktor:} Pracownik terenowy

\textbf{Warunki początkowe:} Pracownik w stanie IN\_PROGRESS, praca skończona

\textbf{Główny przepływ:}

\begin{enumerate}
    \item Pracownik klika ``Zakończ pracę''
    \item System sprawdza formularz raportu (dostosowany do klienta) czy zawiera wszystekie obowiązkowe punkty
    \item Pracownik klika ``Potwierdź''/``Wyślij''
    \item System uploaduje załączniki (jeśli online) i zmienia status na COMPLETED
    \item Powiadomienie ``Zadanie zostało rozliczone''
\end{enumerate}

\textbf{Alternatywne przepływy:}

\begin{itemize}
    \item A1: Offline -- plik raportów zapisany lokalnie, upload po powrocie internetu
    \item A2: Brak wymaganych pól -- komunikat błędu, pracownik uzupełnia
\end{itemize}

\textbf{Warunki końcowe:} Status COMPLETED, raport zapisany na serwerze, zadanie niezmienne

---

% ============================================================================
\section{Scenariusze testowe}
% ============================================================================

\subsection{ST-1: Logowanie online}

\begin{enumerate}
    \item Pracownik uruchamia aplikację
    \item Wpisuje e-mail i hasło
    \item Zaznacza ``Zapamiętaj mnie''
    \item Klika ``Zaloguj''
    \item \checkmark Aplikacja zalogowana, lista zadań pobrana
\end{enumerate}

---

\subsection{ST-2: Auto-logowanie}

\begin{enumerate}
    \item Pracownik w poprzedniej sesji zaznaczył ``Zapamiętaj mnie''
    \item Pracownik uruchamia aplikację (2. raz)
    \item \checkmark Aplikacja automatycznie zalogowana bez wpisywania danych
\end{enumerate}

---

\subsection{ST-3: Praca offline}

\begin{enumerate}
    \item Pracownik zalogowany, ma listę zadań
    \item Wyłącz internet (Airplane Mode)
    \item Pracownik otwiera zadanie
    \item Zmienia status na IN\_PROGRESS
    \item Dodaje checkboxy, notatki, zdjęcia
    \item Kończy zadanie (COMPLETED)
    \item \checkmark Wszystkie operacje działają offline
    \item Włącz internet
    \item \checkmark Zmiany automatycznie synchronizują się
\end{enumerate}

---

\subsection{ST-4: Filtrowanie zadań}

\begin{enumerate}
    \item Pracownik w liście zadań
    \item Klika filtr ``COMPLETED''
    \item \checkmark Lista wyświetla tylko ukończone zadania
    \item Klika filtr ``SUSPENDED''
    \item \checkmark Lista wyświetla tylko wstrzymane zadania
\end{enumerate}

---

\subsection{ST-5: Wstrzymanie z powodem}

\begin{enumerate}
    \item Pracownik w IN\_PROGRESS
    \item Klika ``Wstrzymaj''
    \item Dialog pojawia się
    \item Pracownik NIE wpisuje powodu i klika ``Wstrzymaj''
    \item \checkmark System wyświetla błąd ``Pole obowiązkowe''
    \item Pracownik wpisuje powód
    \item \checkmark Status zmienia się na SUSPENDED
\end{enumerate}

---

\subsection{ST-6: Rozliczenie z zdjęciami (offline + online)}

\begin{enumerate}
    \item Pracownik offline
    \item Kończy pracę, klika ``Zakończ pracę''
    \item Dodaje opisany raport, wymagane zdjęcia
    \item Klika ``Potwierdź''
    \item \checkmark Raport zapisany lokalnie, status COMPLETED
    \item Włącz internet
    \item \checkmark Zdjęcia i raport automatycznie uploadują się
\end{enumerate}

---

% ============================================================================
\section{Model domenowy i słownik pojęć}
% ============================================================================

\subsection{Kluczowe terminy}

\begin{itemize}
    \item \textbf{Task (Zadanie)} -- jednostka pracy przydzielona pracownikowi, zawiera opis, deadline, lokalizację
    \item \textbf{Subtask (Podzadanie)} -- element checklist'u w ramach Task'u
    \item \textbf{Worker (Pracownik)} -- użytkownik aplikacji, osoba terenowa
    \item \textbf{Manager (Kierownik)} -- przełożony pracownika, może przydzielać zadania
    \item \textbf{Administrator} -- osoba z OutsideJob, zarządza użytkownikami i systemem
    \item \textbf{Report (Raport)} -- dokument podsumowujący wykonane zadanie (PDF, JSON, Excel)
    \item \textbf{Status} -- stan zadania: TODO, IN\_TRANSIT, IN\_PROGRESS, SUSPENDED, COMPLETED
    \item \textbf{Synchronizacja} -- dwukierunkowa wymiana danych między urządzeniem a serwerem
    \item \textbf{Offline-first} -- dane najpierw przechowywane lokalnie, synchronizacja gdy internet dostępny
    \item \textbf{Sesja} -- okres zalogowania użytkownika, wygasająca po wylogowaniu lub zalogowaniu użytkownika na innym urządzeniu bądź po 7 dniach braku aktywności
\end{itemize}

---

% ============================================================================
\section{Architektura systemu}
% ============================================================================

\subsection{Komponenty systemu}

\begin{verbatim}
┌────────────────────────────────────┐
│   Android App (OutsideJob)         │
│   ─────────────────────────────    │
│   - UI Layer (Fragments, Compose)  │
│   - ViewModel Layer                │
│   - Repository Layer               │
│   - Local DB (Room/SQLite)         │
│   - WorkManager (Sync)             │
└────────┬─────────────────────────┘
         │ (HTTPS REST API)
         ↓
┌────────────────────────────────────┐
│   Backend Server                   │
│   ─────────────────────────────    │
│   - Authentication Service         │
│   - Task Management Service        │
│   - Report Generation Service      │
│   - Sync Service                   │
│   - PostgreSQL Database            │
│   - File Storage (S3/GCS)          │
└────────────────────────────────────┘
\end{verbatim}

\subsection{Przepływ danych}

\begin{enumerate}
    \item Pracownik uruchamia aplikację
    \item Loguje się (email + password)
    \item Aplikacja pobiera listę Task'ów z serwera
    \item Task'i przechowywane w Room (SQLite)
    \item Pracownik pracuje offline (zmiany lokalnie)
    \item Po powrocie internetu -- WorkManager synchronizuje zmiany
    \item Raport uploadowany do serwera
    \item Serwer generuje raporty w różnych formatach
\end{enumerate}

---

% ============================================================================
\section{Model danych (ERD)}
% ============================================================================

\subsection{Tabela Users}

\begin{lstlisting}[language=SQL]
CREATE TABLE users (
    id BIGSERIAL PRIMARY KEY,
    email VARCHAR(255) UNIQUE NOT NULL,
    password_hash VARCHAR(255) NOT NULL,
    first_name VARCHAR(100),
    last_name VARCHAR(100),
    role VARCHAR(50) DEFAULT 'WORKER',  -- WORKER, MANAGER, ADMIN
    company_id BIGINT NOT NULL,
    created_at TIMESTAMP DEFAULT NOW(),
    updated_at TIMESTAMP DEFAULT NOW(),
    last_login TIMESTAMP,
    FOREIGN KEY (company_id) REFERENCES companies(id)
);
\end{lstlisting}

\subsection{Tabela Tasks}

\begin{lstlisting}[language=SQL]
CREATE TABLE tasks (
    id BIGSERIAL PRIMARY KEY,
    display_id VARCHAR(20) UNIQUE NOT NULL,
    title VARCHAR(255) NOT NULL,
    description TEXT,
    assigned_to BIGINT NOT NULL,
    created_by BIGINT NOT NULL,
    status VARCHAR(50) DEFAULT 'TODO',
    -- TODO, IN_TRANSIT, IN_PROGRESS, SUSPENDED, COMPLETED
    due_date TIMESTAMP,
    location_lat DECIMAL(10, 8),
    location_lng DECIMAL(11, 8),
    created_at TIMESTAMP DEFAULT NOW(),
    updated_at TIMESTAMP DEFAULT NOW(),
    completed_at TIMESTAMP,
    FOREIGN KEY (assigned_to) REFERENCES users(id),
    FOREIGN KEY (created_by) REFERENCES users(id)
);
\end{lstlisting}

\subsection{Tabela Subtasks}

\begin{lstlisting}[language=SQL]
CREATE TABLE subtasks (
    id BIGSERIAL PRIMARY KEY,
    task_id BIGINT NOT NULL,
    title VARCHAR(255) NOT NULL,
    description TEXT,
    type VARCHAR(50),  -- CHECKBOX, TEXT, PHOTO, NOTES
    completed BOOLEAN DEFAULT FALSE,
    completed_at TIMESTAMP,
    FOREIGN KEY (task_id) REFERENCES tasks(id)
);
\end{lstlisting}

\subsection{Tabela Reports}

\begin{lstlisting}[language=SQL]
CREATE TABLE reports (
    id BIGSERIAL PRIMARY KEY,
    task_id BIGINT UNIQUE NOT NULL,
    worker_id BIGINT NOT NULL,
    description TEXT,
    duration_minutes INT,
    signature_base64 TEXT,
    status VARCHAR(50) DEFAULT 'DRAFT',
    -- DRAFT, SUBMITTED, APPROVED, REJECTED
    created_at TIMESTAMP DEFAULT NOW(),
    submitted_at TIMESTAMP,
    FOREIGN KEY (task_id) REFERENCES tasks(id),
    FOREIGN KEY (worker_id) REFERENCES users(id)
);
\end{lstlisting}

\subsection{Tabela Attachments}

\begin{lstlisting}[language=SQL]
CREATE TABLE attachments (
    id BIGSERIAL PRIMARY KEY,
    task_id BIGINT NOT NULL,
    file_path VARCHAR(255),
    file_type VARCHAR(50),  -- PHOTO, PDF, DOCUMENT
    uploaded_at TIMESTAMP DEFAULT NOW(),
    FOREIGN KEY (task_id) REFERENCES tasks(id)
);
\end{lstlisting}

\subsection{Tabela Suspensions}

\begin{lstlisting}[language=SQL]
CREATE TABLE suspensions (
    id BIGSERIAL PRIMARY KEY,
    task_id BIGINT NOT NULL,
    reason TEXT NOT NULL,
    suspended_at TIMESTAMP DEFAULT NOW(),
    resumed_at TIMESTAMP,
    FOREIGN KEY (task_id) REFERENCES tasks(id)
);
\end{lstlisting}

---

% ============================================================================
\section{Interfejs użytkownika (UI/UX)}
% ============================================================================

\subsection{Główne ekrany}

\begin{enumerate}
    \item \textbf{Ekran logowania} -- email, password, checkbox ``Zapamiętaj mnie'', przycisk ``Zaloguj''
    \item \textbf{Lista zadań} -- liste Tasks z filtrami (ALL, TODO, IN\_PROGRESS, COMPLETED, SUSPENDED, OVERDUE), pull-to-refresh
    \item \textbf{Szczegóły zadania} -- informacje, przyciski akcji (Rozpocznij dojazd, Rozpocznij pracę)
    \item \textbf{Ekran pracy} -- checklist podzadań, pola tekstowe, przycisk dodania zdjęcia, Zakończ, Wstrzymaj
    \item \textbf{Podekran pracy} -- szczegółowy ekran z polami w zależności od potrzeb klienta projektowany indywidualnie
    \item \textbf{Navigation Drawer} -- Tasks, + Add Task, Profile, Wyloguj
\end{enumerate}

\subsection{Wytyczne designu}

\begin{itemize}
    \item Material Design 3
    \item Kolory: Primary (Teal #2196F3), Secondary (Gray), Error (Red)
    \item Typography: Roboto (system font)
    \item Spacing: 8px, 16px, 24px
    \item Icons: Material Icons
    \item Accessibility: WCAG 2.1 AA standard
\end{itemize}

---

% ============================================================================
\section{Integracje}
% ============================================================================

\subsection{Firebase (opcjonalnie)}

\begin{itemize}
    \item Push notifications
    \item Crash reporting
    \item Analytics
\end{itemize}


---

% ============================================================================
\section{Plan testów}
% ============================================================================

\subsection{Testy funkcjonalne}

\begin{itemize}
    \item Logowanie (poprawne i niepoprawne dane)
    \item Zapamiętanie hasła
    \item Przeglądanie listy zadań
    \item Filtrowanie
    \item Rozpoczęcie pracy
    \item Wstrzymanie z powodem
    \item Rozliczenie pracy
    \item Synchronizacja danych
\end{itemize}

\subsection{Testy offline}

\begin{itemize}
    \item Praca bez internetu
    \item Synchronizacja po powrocie internetu
    \item Brak konfliktów
\end{itemize}

\subsection{Testy wydajności}

\begin{itemize}
    \item Ładowanie listy 100 zadań
    \item Responsywność UI
    \item Zużycie baterii i pamięci
    \item Upload dużych zdjęć
\end{itemize}

\subsection{Testy bezpieczeństwa}

\begin{itemize}
    \item Szyfrowanie danych w spoczynku i w ruchu
    \item Wygasanie sesji
    \item Logowanie do Keystore
\end{itemize}

---

% ============================================================================
\section{Wytyczne kodowania i konwencje}
% ============================================================================

\subsection{Język programowania}

\begin{itemize}
    \item Kotlin (główny) dla Android App
    \item Java / Kotlin dla Backend'u
    \item TypeScript (opcjonalnie) dla Admin Panel
\end{itemize}

\subsection{Struktura projektu}

\begin{verbatim}
app/
├── data/
│   ├── database/
│   ├── model/
│   ├── remote/ (API calls)
│   └── repository/
├── domain/
│   ├── usecase/
│   └── entity/
├── presentation/
│   ├── ui/
│   │   ├── login/
│   │   ├── tasks/
│   │   ├── task_detail/
│   │   └── work/
│   └── viewmodel/
├── worker/ (sync, background jobs)
└── util/
\end{verbatim}

\subsection{Konwencje nazewnictwa}

\begin{itemize}
    \item Klasy: PascalCase (``TaskDetailFragment'')
    \item Funkcje: camelCase (``fetchTasks'')
    \item Stałe: UPPER\_SNAKE\_CASE (``MAX\_RETRIES'')
    \item XML ids: snake\_case (``btn\_start\_work'')
\end{itemize}

\subsection{Commits}

\begin{itemize}
    \item Format: ``[TYPE] Description''
    \item Typy: feat, fix, refactor, test, docs, chore
    \item Przykład: ``[feat] Add task filtering by status''
\end{itemize}

---

% ============================================================================
\section{Harmonogram wdrożenia}
% ============================================================================

\subsection{Fazy projektu}

\begin{itemize}
    \item \textbf{Faza 1 (Sprint 1-2):} Logowanie, autentykacja, zapamiętanie hasła
    \item \textbf{Faza 2 (Sprint 3-4):} Lista zadań, szczegóły, filtrowanie
    \item \textbf{Faza 3 (Sprint 5-6):} Praca nad zadaniem, checklist, wstrzymanie
    \item \textbf{Faza 4 (Sprint 7-8):} Rozliczenie, raportowanie, upload zdjęć
    \item \textbf{Faza 5 (Sprint 9-10):} Synchronizacja offline-first, WorkManager
    \item \textbf{Faza 6 (Sprint 11-12):} Testy (QA), optymalizacja, deployment
\end{itemize}

\subsection{Kamienie milowe}

\begin{itemize}
    \item Koniec Sprint 4: Alpha (prototyp funkcjonalny)
    \item Koniec Sprint 8: Beta (pełny feature set)
    \item Koniec Sprint 12: Release 1.0.0 (produkcja)
\end{itemize}

---

% ============================================================================
\section{Podsumowanie}
% ============================================================================

Niniejsza specyfikacja stanowi pełną dokumentację wymagań funkcjonalnych i niefunkcjonalnych dla aplikacji OutsideJob. Dokument obejmuje:

\begin{itemize}
    \item 14 głównych wymagań funkcjonalnych (REQ-F)
    \item 7 wymagań niefunkcjonalnych (NFR)
    \item 5 kluczowych przypadków użycia
    \item 6 scenariuszy testowych
    \item Model danych (ERD) z tabelami SQL
    \item Architekturę systemu
    \item Wytyczne UX/UI
    \item Plan testów i wdrażania
\end{itemize}

Dokument stanowi fundament dla całego projektu i powinien być zatwierddzony przez stakeholderów przed przejściem do fazy implementacji.

\subsection{Dalsze kroki}

\begin{enumerate}
    \item Zatwierdzenie specyfikacji przez zespół i klienta
    \item Przygotowanie mockup'ów UI/UX (Figma)
    \item Specyfikacja API REST (OpenAPI/Swagger)
    \item Przygotowanie planu architektonicznego (szczegółowo)
    \item Konfiguracja środowiska (dev, staging, prod)
    \item Rozpoczęcie Sprint 1
\end{enumerate}

\end{document}
